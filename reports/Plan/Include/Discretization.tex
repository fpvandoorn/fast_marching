%Une fois cela fait, on voudra montrer que certains schémas numériques concrets satisfont bien les deux axiomes. 
%Il y a un peu plus de pré-requis mais ça reste raisonnable (pour l'approche la plus simple : un peu d'algèbre linéaire en dimension 2 ou 3, et la manipulation de sommes finies).
%

\subsection{Eulerian schemes}

\subsubsection{General construction}


\begin{proposition}[Combining DDE schemes]
\label{prop:comb_DDE}
Let $F$ and $G$ be DDE (resp.\ causal) schemes over a finite set $X$. The the following schemes are DDE (resp.\ causal) as well
\begin{align*}
	&α F+β G,&
	&\max \{F,G\},&
	&\min \{F,G\}, &
	&η \circ F
%	&v \mapsto F(η \circ v),
\end{align*}
where $α,β ≥ 0$, and $η : \bR \to \bR$ is non-decreasing.
%De plus $FG$ est DDE si $F$ et $G$ sont toujours positifs. Ces propriétés de stabilité valent aussi pour les opérateurs dégénérés elliptiques, dans le cadre continu.
\end{proposition}

\task{Prove this proposition (or a generalization, e.g. to finite sums and maxima).}

\begin{proposition}
Given non-negative coefficients $\alpha : X \times X \times X \to [0,\infty[$, the following scheme is DDE and causal
\begin{align*}
	\forall x \in X \sm X_0, Fu(x) &:= \sum_{y,z \in X} 	\alpha(x,y,z) \max \{0,u(x)-u(y),u(x)-u(z)\}_+^2,\\
	\forall x \in X_0, Fu(x) &:= u(x)-u(x_0).
\end{align*}
\end{proposition}

\task{Prove this proposition, using the previous proposition and building the scheme from simpler ones.}



The classical scheme for the isotropic eikonal equation, on the Cartesian grid $h\bZ^d$, reads 
\begin{equation}
	Fu(x) := \sum_{1 \leq i \leq d} \max \{0, \frac{u(x)-u(x-h e_i)} h, \frac{u(x)-u(x+h e_i)} h\}^2 - 1,
\end{equation}
where $(e_i)_{1 \leq i \leq d}$ denotes the canonical basis. 

\task{Prove that it is consistent with the eikonal equation $\|\nabla u(x)\|^2 -1$, and that it is DDE and causal.}

\subsubsection{Riemannian metrics, and Selling's algorithm}

Consider the scheme on $h\bZ^d$
\begin{equation}
Fu(x) := \sum_{1 \leq i \leq I} \max \rho_i(x) \{0, \frac{u(x)-u(x-h e_i)} h, \frac{u(x)-u(x+h e_i)} h\}^2 - 1,
\end{equation}
where $I$ is an integer, $\rho_i(x) \geq 0$ are weights, $e_i\in \bZ^d \sm \{0\}$ are offsets with integer coordinates. 

\opteasy{Prove that it is consistent with the Riemannian eikonal equation $\|\nabla u(x)\|^2_{D(x)}-1$, where $D(x):=\sum_{1 \leq i \leq I} \rho_i(x) e_i e_i^\top$. }

\optmedium{Prove that Selling's decomposition yields a suitable matrix decomposition.}

\opthard{Prove that Selling's decomposition is uniquely defined.}

\subsection{Semi-Lagrangian schemes}



\subsubsection{Riemannian metrics}

\optmedium{Construction of an acute stencil.}

