
\subsection{General construction}


The objective is to establish basic stabilities for the construction of Eulerian schemes, and to construct the classical numerical scheme for the Isotropic eikonal equation \cite{sethian1996fast}, as well as the scheme from \cite{mirebeau2019riemannian} for the Riemannian eikonal equation (which is the simplest anisotropic case).


\begin{proposition}[Combining DDE schemes]
\label{prop:comb_DDE}
Let $F$ and $G$ be DDE (resp.\ causal) schemes over a (finite) set $X$. The the following schemes are DDE (resp.\ causal) as well
\begin{align*}
	&α F+β G,&
	&\max \{F,G\},&
	&\min \{F,G\}, &
	&η \circ F
%	&v \mapsto F(η \circ v),
\end{align*}
where $α,β ≥ 0$, and $η : \bR \to \bR$ is non-decreasing.
%De plus $FG$ est DDE si $F$ et $G$ sont toujours positifs. Ces propriétés de stabilité valent aussi pour les opérateurs dégénérés elliptiques, dans le cadre continu.
\end{proposition}

\task{Prove this proposition (or a generalization, e.g. to finite sums and maxima).}

\begin{lemma}
Given non-negative coefficients $\alpha : X \times X \times X \to [0,\infty[$, then following scheme is DDE and causal
\begin{align*}
	\forall x \in X \sm X_0, Fu(x) &:= \sum_{y,z \in X} 	\alpha(x,y,z) \max \{0,u(x)-u(y),u(x)-u(z)\}_+^2,\\
	\forall x \in X_0, Fu(x) &:= u(x)-u(x_0).
\end{align*}
\end{lemma}

\task{Prove this proposition, using the previous proposition and building the scheme from simpler ones.}

\begin{remark}
	The assumption that $X$ is finite is used in the proof of the maximum principle, and for convergence analysis purposes.
	
	However it may be annoying when defining finite difference schemes, since in that case we want to work on the Cartesian grid $h \bZ^d$.
	
	It is probably better to keep the definition general (possibly infinite X), and to assume that $X$ is finite only when needed.
\end{remark}


\subsection{The classical isotropic scheme}

The classical scheme for the isotropic eikonal equation, on the Cartesian grid $h\bZ^d$, reads 
\begin{equation}
	Fu(x) := \sum_{1 \leq i \leq d} \max \{0, \frac{u(x)-u(x-h e_i)} h, \frac{u(x)-u(x+h e_i)} h\}^2,
\end{equation}
where $(e_i)_{1 \leq i \leq d}$ denotes the canonical basis. 

\task{Prove that it is consistent with the eikonal equation operator $Fu(x) = \|\nabla u(x)\|^2 + \cO(h)$, and that it is DDE and causal.}



\opteasy{
Prove that numerical schemes of the general form 
\cite[Eq (18)]{mirebeau2019hamiltonian} are DDE and causal.
}



\subsection{Riemannian metrics, and Selling's algorithm}

Consider the scheme on $h\bZ^d$
\begin{equation}
Fu(x) := \sum_{1 \leq i \leq I} \rho_i(x)  \max\{0, \frac{u(x)-u(x-h e_i)} h, \frac{u(x)-u(x+h e_i)} h\}^2 - 1,
\end{equation}
where $I$ is an integer, $\rho_i(x) \geq 0$ are weights, $e_i\in \bZ^d \sm \{0\}$ are offsets with integer coordinates. Such schemes are studied in \cite{mirebeau2019riemannian}.

\task{Prove that it is consistent with the Riemannian eikonal equation $\|\nabla u(x)\|^2_{D(x)}-1$, where $D(x):=\sum_{1 \leq i \leq I} \rho_i(x) e_i e_i^\top$. }

\opteasy{
Prove that numerical schemes of the form \cite[Eq (22)]{mirebeau2019riemannian} are DDE but not causal. 
}



In practice, the matrix $D(x)$ is given, and we want to define suitable weights $\rho_i(x)$.
For that purpose, in dimension $d \in \{2,3\}$, a efficient approach is Selling's decomposition \cite{Selling:1874Algorithm,Conway1992ThreeDimensional}, which is a special case of Voronoi's reduction of quadratic forms \cite{Schurmann2009ComputationalQuadratic}.



\task{
Define the concept of superbase of $\bZ^d$, of $D$-obtuse superbase of $\bZ^d$. 

Prove that Selling's algorithm terminates. 

Prove Selling's formula.

Prove that Selling's decomposition yields a suitable matrix decomposition.

See for instance \cite[Appendix B]{bonnans2022randers}.
}

Selling's decomposition associates to each $D \in S_d^{++}$ a mapping $\rho_D : \bZ^d \sm \{0\} \to [0,\infty[$, finitely supported and such that $\sum_e \rho_D(e) e e^\top = D$. 

In order to study the convergence error of the scheme, one needs to bound the norm of the offsets $e \in \bZ^d$ such that $\rho_D(e) \neq 0$.

\task{
Use the fact that the energy of superbases (denoted $\cE(b)$ in \cite[Proposition B.3]{bonnans2022randers}) decreases along the iterations to prove a basic estimate of $\|e\|$ in terms of $\sqrt{\|D\| \|D^{-1}\|}$ when $\rho_D(e) \neq 0$.
}

\optmedium{
Prove that $\|e\| \leq C \sqrt{\|D\| \|D^{-1}\|}$ when $\rho_D(e) \neq 0$, by using \cite[Proposition 4.8 and Theorem 4.11]{mirebeau2018fast}.
}


\opthard{Prove that Selling's decomposition is unique.}


