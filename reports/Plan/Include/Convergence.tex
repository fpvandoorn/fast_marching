%%Je mentionne comme ouverture possible la preuve que les solutions numériques convergent vers la solution de viscosité de l'EDP eikonale. 
%Les pré-requis deviennent un peu plus sérieux, principalement les fonctions continument différentiables, et le théorème d'Arzela-Ascoli.


The convergence solutions of discretizations of the eikonal equations is studied using the tool of viscosity solution. There are two approaches, the first one is the most general, and the second one provides quantitative estimates. 

\subsection{Viscosity solutions}



\cite{Bornemann:2006FiniteElement}



\task{
Formalize the concept of viscosity solution to a first order HJB PDE. One may follow  \cite{Crandall1984viscosityHJB} and define the notions of super-differential and subdifferential.
}

\task{Prove that the discrete solutions converge to the viscosity solution, for one of the numerical schemes studied. One may follow \cite{Bornemann:2006FiniteElement}.}


\subsection{Quantitative estimates}

The main ingredient is the \emph{doubling of variables argument}.
See for instance \cite[Theorem 2.4]{mirebeau2019riemannian}

\optmedium{Formalize the doubling of variables argument.}

\opthard{Formalize the proof of convergence of solutions to the discretized eikonal equation, with rate $\cO(\sqrt h)$.}